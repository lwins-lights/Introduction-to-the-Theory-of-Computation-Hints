\section{Chapter 1: Regular Languages}

\begin{itemize}
	
	
	\item[\hard 1.45]
	\begin{comment}
	Let $A/B = \{w \ | \ wx \in A \text{ for some } x \in B \}$. Show that if $A$ is regular and $B$ is any language, then $A/B$ is regular. 
	
	\hint
	\end{comment}
	Let $M = (Q, \Sigma, \delta, q_0, F)$ be a DFA such that $L(M) = A$. Consider $ \{ q \in Q \ | \ \exists x \in B,\ \delta(q, x) \in F \} $.
	
	
	\item[\hard 1.56]
	\begin{comment}
	If $A$ is a set of natural numbers and $k$ is a natural number greater than $1$, let
	\[
		B_k(A) = \{ w \ | \ w \text{ is the representation in base $k$ of some number in $A$} \}.
	\]
	Here, we do not allow leading $\mathtt{0}$s in the representation of a number. For example, $B_2(\{3, 5\}) = \{ \mathtt{11}, \mathtt{101} \}$ and $B_3(\{3, 5\}) = \{ \mathtt{10}, \mathtt{12} \}$. Give an example of a set $A$ for which $B_2(A)$ is regular but $B_3(A)$ is not regular. Prove that your example works.
	
	\hint
	\end{comment}
	Let $A = \{2^n \ | \ n \in \mathbb{N} \}$. You may need to do a bit of algebra or number theory.
	
	
	\item[\hard 1.57] 
	\begin{comment}
	If $A$ is any language, let $A_{\frac{1}{2}-}$ be the set of all first halves of strings in $A$ so that
	\[
		A_{\frac{1}{2}-} = \{ x \ | \ \text{for some $y$, $|x| = |y|$ and $xy \in A$} \}.
	\]
	Show that if $A$ is regular, then so is $A_{\frac{1}{2}-}$.
	
	\hint
	\end{comment}
	Let $M$ be a DFA such that $L(M) = A$. Utilizing $M$, design an NFA which accepts $A_{\frac{1}{2}-}$ by guessing something.
	
	
	\item[\hard 1.58] 
	\begin{comment}
	If $A$ is any language, let $A_{\frac{1}{3}-\frac{1}{3}}$ be the set of all strings in $A$ with their middle thirds removed so that	
	\[
		A_{\frac{1}{3}-\frac{1}{3}} = \{ xz \ | \ \text{for some $y$, $|x| = |y| = |z|$ and $xyz \in A$} \}.
	\]
	Show that if $A$ is regular, then $A_{\frac{1}{3}-\frac{1}{3}}$ is not necessarily regular
	
	\hint
	\end{comment}
	Let $A = \mathtt{a}^* \texttt{\#} \mathtt{b}^*$. 
	
	
	\item[\hard 1.59]
	\begin{comment}
	Let $M = (Q, \Sigma, \delta, q_0, F)$ be a DFA and let $h$ be a state of $M$ called its ``home''. A \bm{synchronizing sequence} for $M$ and $h$ is a string $s \in \Sigma^*$ where $\delta(q, s) = h$ for every $q \in Q$. (Here we have extended $\delta$ to strings, so that $\delta(q, s)$ equals the state where $M$ ends up when $M$ starts at state $q$ and reads input $s$.) Say that $M$ is \bm{synchronizable} if it has a synchronizing sequence for some state $h$. Prove that if $M$ is a $k$-state synchronizable DFA, then it has a synchronizing sequence of length at most $k^3$. Can you improve upon this bound?
	
	\hint
	\end{comment}
	Let $Q'$ stands for some subset of $Q$. Try to design a method that finds a relatively short string $w$ such that $|\delta(Q', w)| < |Q'|$, where $\delta(Q', w) = \{ \delta(q, w) \ | \ q \in Q' \}$.
	
	
	\item[\hard 1.63] 
	\begin{comment}
	\begin{itemize}
		\item[a.]
		Let $A$ be an infinite regular language. Prove that $A$ can be split into two infinite disjoint regular subsets.
		\item[b.]
		Let $B$ and $D$ be two languages. Write $B \Subset D$ if $B \subseteq D$ and $D$ contains infinitely many strings that are not in $B$. Show that if $B$ and $D$ are two regular languages where $B \Subset D$, then we can find a regular language $C$ where $B \Subset C \Subset D$.
	\end{itemize}

	\hint
	\end{comment}
	For part a, use the pumping lemma.


	\item[\hard 1.65]
	\begin{comment}
	Prove that for each $n>0$, a language $B_n$ exists where
	\begin{itemize}
		\item[a.]
		$B_n$ is recognizable by an NFA that has $n$ states, and
		\item[b.]
		if $B_n = A_1 \cup \cdots \cup A_k$ for regular languages $A_i$, then at least one of the $A_i$ requires a DFA with exponentially many states.
	\end{itemize}

	\hint
	\end{comment}
	Consider $B_{n+2} = \Sigma^*\mathtt{1}\mathtt{0}^{n}$.
	
	
	\item[\hard 1.67]
	\begin{comment}
	Let the \bm{rotational closure} of language $A$ be $RC(A) = \{ yx \ | \ xy \in A \}$.
	\begin{itemize}
		\item[a.]
		Show that for any language $A$, we have $RC(A) = RC(RC(A))$.
		\item[b.]
		Show that the class of regular languages is closed under rotational closure.
	\end{itemize}
	\hint
	\end{comment}
	For part b, let $M$ be a DFA such that $L(M) = A$. Utilizing $M$, design an NFA which accepts $RC(A)$ by guessing something.
	
	
	\item[\hard 1.68] 
	\begin{comment}
	In the traditional method for cutting a deck of playing cards, the deck is arbitrarily
	split two parts, which are exchanged before reassembling the deck. In a more
	complex cut, called Scarne’s cut, the deck is broken into three parts and the middle
	part in placed first in the reassembly. We’ll take Scarne’s cut as the inspiration for
	an operation on languages. For a language $A$, let $CUT(A) = \{ yxz \ | \ xyz \in A \}$.
	\begin{itemize}
		\item[a.]
		Exhibit a language $B$ for which $CUT(B) \neq CUT(CUT(B))$.
		\item[b.]
		Show that the class of regular languages is closed under $CUT$.
	\end{itemize}
	\hint
	\end{comment}
	Solve problem 1.67 first.

	
\end{itemize}