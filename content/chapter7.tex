\section{Time Complexity}

\begin{itemize}
	
	%		7.13		%
	\item[7.13]
	$a^b = (a^{\lfloor b/2 \rfloor})^2 \cdot a^{b \bmod 2}$, where $a, b \in \mathbb{N}$.
	
	%		7.14		%
	\item[7.14]
	$q^t = (q^{\lfloor t/2 \rfloor})^2 \cdot q^{t \bmod 2}$, where $q \in S_k$ and $t \in \mathbb{N}$.
	
	%		7.15		%
	\item[7.15] 
	The hint in the textbook is sufficient.
	
	%		7.16		%
	\item[7.16]
	\Omit
	
	%		7.17		%
	\item[7.17]
	Use dynamic programming. Denote $dp[i][j] = \boldsymbol{1}\{ \Bra{\{x_1, \dots, x_i\}, j} \in \IT{SUBSET-SUM} \}$, where $\alpha \implies \boldsymbol{1}\{\alpha\} = 1$ and $\neg \alpha \implies \boldsymbol{1}\{\alpha\} = 0$.
	
	%		7.18		%
	\item[7.18]
	First $A \in \Poly = \NPoly$. Then since there exist $x \in A$ and $y \notin A$, for an arbitrary language $B \in \NPoly = \Poly$, $f$ defined as follows is polynomial time computable.
	$$
		f(w) = 
		\left\{
			\begin{array}{ll}
				x, & w \in B \\
				y, & w \notin B
			\end{array}
		\right.
	$$
	
	%		7.19		%
	\item[\Star 7.19]
	Let the certificate of $q \in \mathbb{P}$ consist of
	\begin{itemize}
		\item $g \in \mathbb{Z}_m^*$ such that $g^{m-1}=1$ and $g^{(m-1)/q} \neq 1$ for all prime $q \ | \ m-1$,
		\item the standard factorization of $m-1 = \prod q_i^{r_i}$,
		\item certificates of $q_i \in \mathbb{P}$.
	\end{itemize} 

	%		7.20		%
	\item[7.20]
	It follows from the result of Problem 7.18.
	
	%		7.21		%
	\item[7.21]
	\begin{itemize}
		\item[a.] Modify the proof of $\textit{PATH} \in \Poly$.
		\item[b.] It is in $\NPoly$ evidently. For the other side, reduce from $\IT{UHAMPATH}$ by setting $k$ to the amount of nodes in $G$ minus $1$.
	\end{itemize}

	%		7.22		%
	\item[7.22]
	It is in $\NPoly$ evidently. For the other side, reduce from $\IT{SAT}$. In order to determine whether $\Bra{\phi} \in \IT{SAT}$, construct $\phi' = \phi \,\wedge\,(z \vee \overline{z})$.
	
	%		7.23		%
	\item[7.23]
	\Omit
	
	%               7.25            %
	\item[7.25]
	If there is no unitary negated clause in $ \phi $, then $ \phi $ is satisfiable since we can just assign value 1 to all variables. However, if $ \phi $ contains some clauses in the form $ (\overline{x_i}) $, then this implies that the only possible way to make $ \phi $ true is to let those $ x_i $ equal 0. So $ \phi $ can be reduced to some shorter formula repeatedly.
	
	%		7.36		%
	\item[\Star 7.36] 
	It is in $\NPoly$ evidently. For the other side, reduce from $\IT{3SAT}$. If there is no limitation on $\Sigma$, let the following \DFA\ correspond to an assignment to variables $x_1, x_2, \dots, x_n$ in 3cnf-formula $\phi$. You may need some extra states and transitions.
	$$
		\xymatrix@ur@R=2pc@C=2pc{
			*+[o][F-]{\RM{S}} \ar@(l,l)[]^{} \ar@/^/[r]^{\{ \TT{x}_i | x_i = 1 \}} \ar@/_/[d]_{\{ \TT{x}_i | x_i = 0 \}} &
			*+[o][F-]{T} \ar^{\TT{t}}[d] \ar^{\TT{f}}[l] \\
			*+[o][F-]{F} \ar_{\TT{f}}[r] \ar_{\TT{t}}[u] & 
			*+[o][F-]{A} \ar@(ur,rd)[]^{\Sigma}
		}
	$$
	Once you solved the problem without regard to the limitation on $\Sigma$, based on your solution, consider how to build a reduction where $\Sigma = \{ \TT{0}, \TT{1} \}$.
	
	%		7.37		%
	\item[7.37]
	Using the computation history as certificate we easily obtain $U \in \NPoly$. And it is easy to show that $\IT{3SAT} \LeqP U$, by designing an \NTM\ $M$, which accepts $\Bra{\phi}$ in polynomial time on at least one branch if and only if $\Bra{\phi} \in \IT{3SAT}$.
	
	%		7.38		%
	\item[\Star 7.38]
	Let $\phi(x/t)$ denote the Boolean formula $\phi$ after replacing every existance of $x$ in $\phi$ with $t$. Suppose there are $n$ variables $x_1, \dots, x_n$ in $\phi$. $\Bra{\phi} \in \IT{SAT} \implies \Bra{\phi(x_1/0)} \in \IT{SAT} \text{ or } \Bra{\phi(x_1/1)} \in \IT{SAT}$, so we can directly assign $0$ or $1$ to $x_1$. Recursively assigning $x_2, \dots, x_n$ in this way we have done.
	
	%		7.39		%
	\item[\Star 7.39]
	Construct language $L = \{ \Bra{n, x, y} \ | \ \text{$n$ has a nontrivial factor in interval $[x, y]$} \}$, which is apparently in $\NPoly = \Poly$. Then refer to the solution to Problem 7.38.
	
	%		7.40		%
	\item[\Star 7.40] 
	\Omit
	
	%		7.41		%
	\item[7.41]
	Trivial.
	
	%		7.42		%
	\item[\Star 7.42]
	For b), note that the complement graph of $G$ induces an equivalence relation $\sim$ on $Q$ ($[q]$ is exactly the equivalence class under $\sim$ including $q$), which has much to do with $\equiv_{L(M)}$ defined in Problem 1.51.
	
	%		7.43		%
	\item[7.43]
	Here is a sample for $\phi = (x_1 \vee \overline{x_2} \vee x_3) \wedge (x_2 \vee \overline{x_3})$.
	$$
	\xymatrix@R=2pc@C=2pc{
		*+<1pc>[o][F-]{S} \ar@(l,l)[]^{} \ar[r]^{\epsilon} \ar[rd]_{\epsilon} & 
		*+[o][F-]{A_0} \ar[r]^{\TT{0}} & 
		*+[o][F-]{A_1} \ar[r]^{\TT{1}} & 
		*+[o][F-]{A_2} \ar[r]^{\TT{0}} &
		*+[o][F-]{A_3} \ar[rd]^{\epsilon} \\ & 
		*+[o][F-]{B_0} \ar[r]^{\TT{0} \cup \TT{1}} & 
		*+[o][F-]{B_1} \ar[r]^{\TT{0}} & 
		*+[o][F-]{B_2} \ar[r]^{\TT{1}} &
		*+[o][F-]{B_3} \ar[r]^{\epsilon} &
		*+<1pc>[o][F=]{T}
		% *+[o][F-]{\RM{S}} \ar@/^/[r]^{\{ \TT{x}_i | x_i = 1 \}} \ar@/_/[d]_{\{ \TT{x}_i | x_i = 0 \}} &
		% *+[o][F-]{T} \ar^{\TT{t}}[d] \ar^{\TT{f}}[l] \\
		% *+[o][F-]{F} \ar_{\TT{f}}[r] \ar_{\TT{t}}[u] & 
		% *+[o][F-]{A} \ar@(ur,rd)[]^{\Sigma}
	}
	$$
	And $\Bra{\phi} \notin \IT{SAT} \iff $ the equivalent minimal \NFA\ is a trivial one.
	
	%		7.44		%
	\item[\Star 7.44]
	It is a classical problem. See \url{https://en.wikipedia.org/wiki/2-satisfiability} or refer to Problem 7.51.
	
	%		7.45		%
	\item[7.45]
	Trivial.
	
	%		7.46		%
	\item[7.46]
	Since $P$ is closed under complement, we only need to show that $\overline{\IT{MIN-FORMULA}} \in \NPoly = \Poly$. It is easy to do because $\IT{SAT} \in \NPoly = \Poly$, and then the certicifate for $\Bra{\phi} \in \overline{\IT{MIN-FORMULA}}$ can be $\Bra{\phi'}$ such that $|\phi'| < |\phi|$ and they are equivalent.
	
	%		7.47		%
	\item[7.47]
	$Z = \{ \Bra{G_1, k_1, G_2, k_2} \ | \ \Bra{G_1, k_1} \in \IT{CLIQUE} \} - \{ \Bra{G_1, k_1, G_2, k_2} \ | \ \Bra{G_2, k_2} \in \IT{CLIQUE} \} $.
	
	%		7.48		%
	\item[\Star 7.48]
	Obviously $\IT{MAX-CLIQUE} \in \RM{DP}$. Suppose there is a graph $G = (V, E)$ with $V = \{v_1, v_2, \dots, v_n \}$. Denote $[n] = \{1, 2, \dots, n\}$. Let $G_+ = (V_+, E_+)$, where
	$$
		\left\{
			\begin{array}{l}
				\displaystyle
				V_+ = \{ (i, v_j) \ | \ i \in [n + 1],\ v_j \in V \} \cup \{ (i, \alpha) \ | \ i \in [n+1] - [k] \} \\
				E_+ = \{ \{ (i, u), (j, v) \} \ | \ i \neq j,\ \{u, v\} \in E \} \cup \{ \{ (i, \alpha), (j, v) \} \ | \ i \neq j,\ v \in V \cup \{\alpha\} \}
			\end{array}
		\right.
	$$
	Then $\Bra{G, k} \in \IT{CLIQUE} \iff \Bra{G_+, n+1} \in \IT{MAX-CLIQUE}$. Let $G_-$ consist of $G_+$ and a $n$-clique (i.e., complete graph $K_n$). Then $\Bra{G, k} \notin \IT{CLIQUE} \iff \Bra{G_-, n} \in \IT{MAX-CLIQUE}$. Try to build a reduction by taking advantage of $G_+$ and $G_-$.
	% And we show that $Z \leq_{\RM{T}}^{\Poly} \IT{MAX-CLIQUE}$, where $Z$ is defined in Problem 7.47. To determine whether $\Bra{G_1, k_1, G_2, k_2} \in Z$, we first use the oracle for $\IT{MAX-CLIQUE}$ repeatedly to find the size of largest clique in $G_1$ and $G_2$, denoted as $l_1$ and $l_2$ respectively. Then $\Bra{G_1, k_1, G_2, k_2} \in Z \iff k_1 \leq l_1,\ k_2 > l_2$.
	
	%		7.49		%
	\item[\Star 7.49]
	\Empty
	
	%		7.50		%
	\item[\Star 7.50] 
	$\overline{EQ_{\textsf{SF-REX}}} = \{ \Bra{R, S} \ | \ \exists c,\ c \in L(R) \oplus c \in L(S) \}$. Determining whether $c \in L(R)$ can be achieved in (deterministic) polynomial time by constructing a corresponding \NFA . Note that any $c \in L(R)$ for a star-free \REX\ satisfies $|c| \leq \RM{Poly}(|R|)$.
	
	%		7.51		%
	\item[\Star 7.51] 
	Trivial.
	
	%		7.52		%
	\item[\Star 7.52]
	\Empty
	
	%		7.53		%
	\item[\Star 7.53]
	\Empty
	
	%		7.54		%
	\item[7.54]
	\Empty
	
\end{itemize}
