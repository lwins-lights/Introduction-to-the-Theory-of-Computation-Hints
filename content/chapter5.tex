\section{Reducibility}

\begin{itemize}
	
	%		5.9			%
	\item[5.9]
	Reduce from $A_{\TM}$. To determine whether \TM\ $M$ accepts $w$, construct \TM\ $N$ which always accepts $\TT{01}$ but accepts $\TT{10}$ if and only if $M$ accepts $w$.
	
	%		5.10		%
	\item[5.10]
	\Omit
	
	%		5.11		%
	\item[5.11]
	\Omit
	
	%		5.12		%
	\item[5.12]
	Reduce from $E_{\TM}$. To determine whether \TM\ $M$ accepts nothing, construct \TM\ $N$ which simulates $M$ on $N$'s own input $w$ but never writes a blank symbol over a nonblank symbol unless when $M$ accepts.
	
	%		5.13		%
	\item[5.13]
	Reduce from $E_{\TM}$. To determine whether \TM\ $M$ accepts nothing, construct \TM\ $N$ which simulates $M$ on $N$'s own input $w$ and obviously has no useless state except the one $q_{\RM{accept}}$.
	
	%		5.14		%
	\item[5.14]
	Reduce from $A_{\TM}$. To determine whether \TM\ $M$ accepts $w$, construct \TM\ $N$ which simulates $M$ on $N$'s own input $w$ but never attempts to move its head left when its head is on the left-most tape cell unless when $M$ accepts.
	
	%		5.15		%
	\item[5.15]
	Let $M'$ be $M$ after modifying all its transitions $\delta(q_i,a) = (q_{\RM{accept}},b,X)$ to $\delta(q_i,a) = (q_{\RM{reject}},b,X)$, and then modifying all $\delta(q_i,a) = (q_j,b,\RM{L})$ to $\delta(q_i,a) = (q_{\RM{accept}},b,\RM{R})$. The problem is now reduced to checking whether $M'$, as a \emph{\TM\ with stay put instead of left} described in Problem 3.13, accepts $w$. It is easy since $L(M')$ is regular by the solution to Problem 3.13.
	
	%		5.16		%
	\item[5.16]
	Suppose for the sake of contradiction that $\IT{BB}$ is computable. Then obviously there exists a \TM\ $M$ having $k$ states (let $k$ be sufficiently large), which writes $\IT{BB}(n) + 1$ \TT{1}s on the tape, when given input $\Bra{n}$. Further, using $M$ we can construct a series of \TM s $M_n$ having exactly $k+n/2$ states for large $n$, which writes $\IT{BB}(n) + 1$ \TT{1}s on the tape when started with a blank tape. However, then $M_{2k}$, as a $2k$-state \TM , would write $\IT{BB}(2k) + 1$ \TT{1}s. Absurd.
	
	%		5.17		%
	\item[5.17]
	Obviously, in this case a PCP instance $P$ always has a match unless all dominos in $P$ have longer top strings, or they all have longer bottom strings.
	
	%		5.18		%
	\item[5.18]
	Reduce from $\IT{PCP}$, since any string with finite alphabet can be encoded to a binary one.
	
	%		5.19		%
	\item[5.19]
	Trivial.
	
	%		5.20		%
	\item[5.20]
	We can encode any language to the one over the unary alphabet.
	
	%		5.21		%
	\item[5.21]
	The hint in the textbook is sufficient.
	
	%		5.22		%
	\item[5.22]
	\begin{itemize}
		\item[$\Leftarrow$:] Trivial.
		\item[$\Rightarrow$:] Let $f(x) = \Bra{M, x}$, where \TM\ $M$ is $A$'s recognizer. 
	\end{itemize}

	%		5.23		%
	\item[5.23]
	\begin{itemize}
		\item[$\Leftarrow$:] Trivial, since $\TT{0}^*\TT{1}^*$ is surely decidable.
		\item[$\Rightarrow$:] Suppose there is an $A$'s decider, then $f$ defined as follows is computable.
		\[
			f(x) = 
			\left\{
				\begin{array}{ll}
					\TT{01}, & x \in A \\
					\TT{10}, & x \notin A
				\end{array}
			\right.
		\]
	\end{itemize}

	%		5.24		%
	\item[5.24]
	It immediately follows from $A_{\TM}, \overline{A_{\TM}} \Leqm J$.
	
	%		5.25		%
	\item[5.25]
	$\overline{E_{\TM}} \Leqm A_{\TM}$ by Problem 5.22 and it is well known that $A_{\TM} \Leqm E_{\TM}$. By the way, it is not difficult to construct an undecidable language $B$ such that $B =_{\RM{m}} \overline{B}$.
	
	%		5.26		%
	\item[5.26]
	The idea is the same as how we deal with $A_{\LBA}$ and $E_{\LBA}$.
	
	%		5.27		%
	\item[5.27]
	Prove that $A_{\TM} \Leqm E_{\textsf{2DIM-DFA}} \Leqm EQ_{\textsf{2DIM-DFA}}$, in which the former reduction can be done by computation history method.
	
	%		5.28		%
	\item[\Star 5.28]
	\Omit
	
	%		5.29		%
	\item[5.29]
	The case $P$ is not nontrivial is trivial. As for the second condition, let $P = \{ \Bra{M} \ | \ \TM\ M \text{ has $100$ states} \}$.
	
	%		5.30		%
	\item[5.30]
	Trivial.
	
	%		5.31		%
	\item[5.31]
	Let $M$ be a \TM\ which on input $\Bra{x}$ ($x \in \mathbb{Z}^+$) calculates $x, f(x), f(f(x)), \dots$ until it finds some $f^{(n)}(x)	= 1$, and then accepts. Let $N$ be a \TM\ uses $H$ to calculate whether $\Bra{M, x} \in A_{\TM}$ for $x = 1, 2, \dots$ in order, until it finds some $\Bra{M, x} \notin A_{\TM}$, and then accepts. We have the positive answer to the $3x + 1$ problem if and only if $\Bra{N, \TT{0}} \notin A_{\TM}$. 
	
	%		5.32		%
	\item[5.32]
	\begin{itemize}
		\item[a.] As hinted, reduce from $\IT{PCP}$.
		\item[b.] Reduce from $\IT{OVERLAP}_{\CFG}$ by constructing a grammar whose rules are $G$'s and $H$'s rules and $S \to S_G \TT{\$} \ | \ S_H \TT{\$\$}$, where $S_G$ and $S_H$ are $G$'s and $H$'s start variables.
	\end{itemize}

	%		5.33		%
	\item[5.33]
	Refer to the proof of undecidability of $ALL_{\CFG}$. Let $w = \TT{\#} C_1 \TT{\#} C_3 \TT{\#} C_5 \TT{\#} \cdots \TT{\#} C_6^\mathcal{R} \TT{\#} C_4^\mathcal{R} \TT{\#} C_2^\mathcal{R} \TT{\#}$.
	
	%		5.34		%
	\item[5.34]
	Reduce from $A_{\TM}$. To determine whether \TM\ $N$ accepts $w$, construct \TM\ $M$ which simulates $N$ on $M$'s own input $w$ but never modifies the portion of the
	tape that contains the input $w$ unless when $N$ accepts.
	
	%		5.35		%
	\item[5.35]
	\begin{itemize}
		\item[a.] Just enumerate $w$.
		\item[b.] Reduce from $ALL_{\CFG}$. In order to determine wether $L(G) = \Sigma^*$, construct a grammar whose rules are $G$'s rules and $S \to S_G \ | \ T;\ T \to aT \ | \ \epsilon \ (a \in \Sigma)$, where $S_G$ are $G$'s start variable.
	\end{itemize}
	
	%		5.36		%
	\item[\Star 5.36]
	See \url{https://cstheory.stackexchange.com/q/39407/46760} for two different solutions.
	
\end{itemize}