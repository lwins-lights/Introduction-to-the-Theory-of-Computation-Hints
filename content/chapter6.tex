\section{Advanced Topics in Computability Theory}

\begin{itemize}
	
	%		6.6			%
	\item[6.6]
	Let $M = P_{\Bra{N}}$ and $N$ print $q(\Bra{N}) = \Bra{M}$.
	
	%		6.7			%
	\item[6.7]
	A \TM\ that always loops.
	
	%		6.8			%
	\item[\Star 6.8]
	Suppose for the sake of contradiction that $f$ is a reduction from $EQ_{\TM}$ to $\overline{EQ_{\TM}}$. It is easy to generalize the fixed-point version of the recursion theorem to find $f(\Bra{M, N}) = \Bra{M', N'}$ such that $M, N$ simulate $M', N'$ respectively. Then $\Bra{M, N} \in EQ_{\TM} \iff \Bra{M', N'} \in \overline{EQ_{\TM}} \iff \Bra{M, N} \in \overline{EQ_{\TM}}$. Absurd.
	
	%		6.9			%
	\item[6.9]
	\Omit
	
	%		6.10		%
	\item[6.10]
	\Omit
	
	%		6.11		%
	\item[\Star 6.11]
	$(\mathbb{R}, =, <)$.
	
	%		6.12		%
	\item[6.12]
	\Omit
	
	%		6.13		%
	\item[6.13]
	Since $\mathbb{Z}_m$ is finite, any sentence in the language of $\mathcal{F}_m$ can be decided by brute-force checking.
	
	%		6.14		%
	\item[6.14]
	Let $J = \TT{0}A \cup \TT{1}B$.
	
	%		6.15		%
	\item[6.15]
	Let $B = A_{\TM^A} = \{ \Bra{M^{A}, w} \ | \ \text{$M^A$ accepts $w$} \}$. Then apply any classical method used in proving undecidablity of $A_{\TM}$.
	
	%		6.16		%
	\item[\Star 6.16]
	\Attr{Kleene--Post} For convenience let any language $L$ be a subset of $\mathbb{N}$ instead of $\Sigma^*$. Denote $\{0, 1, \dots, m\}$ by $[m]$. Define 
	$$
		\mathcal{L}_m(A) = \{ L \subseteq \mathbb{N} \ | \ L \cap [m] = A\} \quad (A \subseteq [m])
	$$
	Let $M_0, M_1, M_2, \dots$ be all possible oracle \TM s. We will give two series of families of languages $\mathcal{A}_0 \supseteq \mathcal{A}_1 \supseteq \mathcal{A}_2 \supseteq \cdots$ and $\mathcal{B}_0 \supseteq \mathcal{B}_1 \supseteq \mathcal{B}_2 \supseteq \cdots$ such that
	$$
		\forall A \in \mathcal{A}_n, B \in \mathcal{B}_n,\ M_n^A \text{ is not $B$'s decider} \text{ and } M_n^B \text{ is not $A$'s decider}.
	$$
	Then taking arbitrary $A \in \mathcal{A} = \bigcap_{n \in \mathbb{N}} \mathcal{A}_n$ and $B \in \mathcal{B} = \bigcap_{n \in \mathbb{N}} \mathcal{B}_n$ we have $A \nleq_{\RM{T}} B$ and $B \nleq_{\RM{T}} A$. We build them by induction. Given $\mathcal{A}_{n-1} = \mathcal{L}_m(X)$ and $\mathcal{B}_{n-1} = \mathcal{L}_m(Y)$, we first find $\mathcal{A}'_n \subseteq \mathcal{A}_{n-1}$ and $\mathcal{B}'_n \subseteq \mathcal{B}_{n-1}$ such that $\forall A \in \mathcal{A}'_n, B \in \mathcal{B}'_n,\ M_n^A \text{ is not $B$'s decider}$.
	\begin{itemize}
		\item If there is no $A \in \mathcal{A}_{n-1}$ such that $M_n^A$ is a decider, let $\mathcal{A}'_n = \mathcal{A}_{n-1}$ and $\mathcal{B}'_n = \mathcal{B}_{n-1}$.
		\item Suppose $M_n^A$ is a decider with some $A \in \mathcal{A}_{n-1}$, then there exists an $m'>m$ such that 
		$$
			\forall A' \in \mathcal{L}_{m'}(A \cap [m']),\ m+1 \in L(M_n^A) \iff m+1 \in L(M_n^{A'}).
		$$
		Then, let $\mathcal{A}'_n = \mathcal{L}_{m'}(A \cap [m'])$, $\mathcal{B}'_n = \mathcal{L}_{m'}(Y)$ or $\mathcal{L}_{m'}(Y \cup \{m+1\})$, depending on whether $m+1 \in L(M_n^A)$.
	\end{itemize}
	The same method can be also used to find $\mathcal{A}_n \subseteq \mathcal{A}'_{n}$ and $\mathcal{B}_n \subseteq \mathcal{B}'_{n}$ such that $\forall A \in \mathcal{A}_n, B \in \mathcal{B}_n,\ M_n^B $ is not $A$'s decider.
	
	%		6.17		%
	\item[\Star 6.17]
	Let 
	\begin{align*}
		A = \{ \Bra{M, w} \ | \ \text{\TM\ $M$ on input $w$ halts with \TT{0} on its tape} \} \\
		B = \{ \Bra{M, w} \ | \ \text{\TM\ $M$ on input $w$ halts with \TT{1} on its tape} \}
	\end{align*}
	If there is a $C$'s decider $N$, we can construct \TM\ $M$ which on input $w$ first run $N$ on $\Bra{M, w}$ to know that $M$ would not halt with $x \in \{ \TT{0}, \TT{1} \}$ on $M$'s tape, and then violates it.
	
	%		6.18		%
	\item[6.18]
	Suppose $L(M) \neq L(N)$, we can enumerate $x$ to find one such that $\Bra{M, x} \in A_{\TM} \oplus \Bra{N, x} \in A_{\TM}$ holds, where $\oplus$ means exclusive or.
	
	%		6.19		%
	\item[6.19]
	$ |\{ L(M^A) \ | \ \text{$M^A$ is an oracle \TM} \}| \leq |\{ M^A \ | \ \text{$M^A$ is an oracle \TM} \}| \leq \aleph_0 < 2^{\aleph_0} = |\{ L \ | \ \text{$L$ is a language} \}|$.
	
	%		6.20		%
	\item[6.20]
	Let $M$ be $PCP$'s recognizer. Then check whether $\Bra{M, \Bra{P}} \in A_{\TM}$ to know if instance $P$ has a match.
	
	%		6.21		%
	\item[6.21]
	Since $\K(x) \leq |x| + c$, we can check all possible minimal description $\Bra{M, w}$ to see if $M$ on input $w$ halts with $x$ on its tape by simulating $M$, where ``possible'' means that $|\Bra{M, w}| < |x| + c$ and $\Bra{M, w} \in A_{\TM}$.
	
	%		6.22		%
	\item[6.22]
	Trivial.
	
	%		6.23		%
	\item[6.23]
	Reduce from Problem 6.24.
	
	%		6.24		%
	\item[6.24]
	Reduce from Problem 6.25.
	
	%		6.25		%
	\item[6.25]
	If not, there is an enumerator $E$ which would print infinite many incompressible strings one by one: $s_1, s_2, \dots$ By using $E$ we can construct \TM\ $M$, which prints an incompressible string $s_i$, such that $|s_i| > |\Bra{M, \TT{0}}|$, on its tape. Then we have $\K(s_i) \leq |\Bra{M, \TT{0}}| < |s_i|$. Absurd.
	
	%		6.26		%
	\item[\Star 6.26] 
	\Sol{1}
	Suppose for the sake of contradiction that $\K(xy) \leq \K(x) + \K(y) + c$ always holds. Define
	$$
		f_n = \sum_{|x|=n} {2^{-\K(x)}}.
	$$
	Then $\K(xy) \leq \K(x) + \K(y) + c \implies f_{n+m} \geq 2^{-c} f_n f_m \implies f_{k n} \geq (2^{-c} f_n)^{k}$. On the other hand, Corollary 6.30 implies $f_n \leq n + 1$. Therefore, 
	$$
		f_n \leq 2^c \sqrt[k]{f_{k n}} \leq 2^c \sqrt[k]{kn + 1}.
	$$
	Letting $k \to +\infty$ we obtain that $f_n \leq 2^c$ for all $n$. However, there is a \TM\ $M$ which on input $\Bra{p, q}$ ($p, q \in \mathbb{N}$ and $q < 2^{2^p}$) halts with $r(p,q)$, a $2^p$-bits binary representation of $q$, on its tape. So $\K(r(p,q)) \leq 2 \log_2 p + \log_2 q + d$ with some constant $d$. Then, if $n = 2^p$ for some large $p$,
	$$
		f_{n} = \sum_{|x|=n} {2^{-\K(x)}} \geq \sum_{q < 2^n} {2^{-\K(r(p,q))}} \geq 2^{-2 \log_2 p - d} \sum_{q < 2^n} \frac{1}{q} \geq \frac{n}{2^d (\log_2 n)^2},
	$$
	which apparently contradicts with $f_n \leq 2^c$.
	
	\Sol{2}
	$ \forall c $, the following process of choosing $ x $ and $ y $ constructs the inequality directly.
	
	First, select an incompressible string $ w $ with length $ n $. Denote its prefix string of length $ \frac{1}{2} \log n $ as $ u $. Let number $ p $ equal the value of $ 1u $ when perceived as a  binary number.
	
	Then, divide $ w $ into two substrings $ w=xy $, where $ \left| x \right| = \frac{1}{2} \log n + p $ and $ \left| y \right| = n - \left| x \right| $ (since $ p < 2^{\frac{1}{2} \log n + 1} = 2 \sqrt{n}$, $ \left| y \right| > 0 $ is a valid string as long as $ n $ is large enough).
	
	We observe that since $ w $ is incompressible, $ K(xy) \ge n $. Also, $ K(y) \le \left| y \right| + c_1 $ for some constant $ c_1 $ independent of $ n $.
	
	We claim that $ K(x) \le p + c_2 $ for some constant $ c_2 $ independent of $ n $. In fact, the following Turing machine $ M $ generates $ x $ when the input string is the tail string of $ x $ with length $ p $:
	
	M="With input string $ z $:
	
	~~~~~~Calculate $ q = \left| z \right| $ and write $ q $ in binary representation on the tape;
	
	~~~~~~Remove the leading character 1 of $ q $, append $ q $ with $ z $, and output them together."
	
	Now, with all the above claims, we get $ K(xy) - \left( K(x) + K(y) + c \right) \ge \frac{1}{2} \log n - \left( c_1 + c_2 + c \right) > 0 $ as long as $ n $ is large enough.
	
	%		6.27		%
	\item[6.27]
	Show that $\overline{\IT{HALT}_{\TM}} \Leqm S, \overline{S}$.
	
	%		6.28		%
	\item[6.28]
	\begin{enumerate}
		\item[a.] $x = 0 \iff \forall y,\ x+y=y$
		\item[b.] $x = 1 \iff \forall y,\ y=0 \wedge x+y=1$ 
		\item[c.] $x = y \iff \forall z,\ z=0 \wedge x+z=y$
		\item[d.] $x < y \iff \exists z,\ \neg(z = 0) \wedge x + z = y$
	\end{enumerate}
	
\end{itemize}
